%
% File acl2019.tex
%
%% Based on the style files for ACL 2018, NAACL 2018/19, which were
%% Based on the style files for ACL-2015, with some improvements
%%  taken from the NAACL-2016 style
%% Based on the style files for ACL-2014, which were, in turn,
%% based on ACL-2013, ACL-2012, ACL-2011, ACL-2010, ACL-IJCNLP-2009,
%% EACL-2009, IJCNLP-2008...
%% Based on the style files for EACL 2006 by 
%%e.agirre@ehu.es or Sergi.Balari@uab.es
%% and that of ACL 08 by Joakim Nivre and Noah Smith

\documentclass[11pt,a4paper]{article}
\usepackage[hyperref]{acl2019}
\usepackage{times}
\usepackage{latexsym}

\usepackage{url}

%\aclfinalcopy % Uncomment this line for the final submission
%\def\aclpaperid{***} %  Enter the acl Paper ID here

%\setlength\titlebox{5cm}
% You can expand the titlebox if you need extra space
% to show all the authors. Please do not make the titlebox
% smaller than 5cm (the original size); we will check this
% in the camera-ready version and ask you to change it back.

\newcommand\BibTeX{B\textsc{ib}\TeX}

\title{Project Proposal}

\author{asdf \\
  Affiliation / Address line 1 \\
  Affiliation / Address line 2 \\
  Affiliation / Address line 3 \\
  \texttt{emdsdl@domain.com} \\\And
  Second Author \\
  Affiliation / Address line 1 \\
  Affiliation / Address line 2 \\
  Affiliation / Address line 3 \\
  \texttt{email@domain} \\}

\date{}

\begin{document}
\maketitle
\begin{abstract}
  This document contains the proposals for preparing the Natual Language Processing course for Spring 2019. 
\end{abstract}

\section{Proposal 1: Conflict Identification}

\subsection{Context}
Online learning become increasingly vital in the current educational system. Due to the advantages in flexibility and convenience, students in both higher education and in K-12 settings grow demand in online learning. A discussion board is used as the main shared space to collaboratively construct knowledge. However, due to the lack of social presence and text-based heavy in the discussion board, students experience difficulties in online communication. Conflicts in communication can occur and be detrimental to learning processes. Instructors need to monitor and regulate conflicts in order to facilitate effective learning. Therefore, we attempt  to develop a tool that helps instructors identify conflicts in the vast body of text such as online discussion board so that instructors could intervene and manage conflicts in a timely manner. 

\subsection{Label}
Conflicting sentiments





\end{document}
